\documentclass[]{article}
\setlength{\pdfpagewidth}{8.5in} \setlength{\pdfpageheight}{11in}
\begin{document}
\begin{table}[htbp]
\centering
\begin{tabular}{lcc} \hline\hline
& \multicolumn{2}{c}{\textit{Dependent Variable: Homicide Rate per 100,000 inhabitants}} \\
\cline{2-3} \\
& (1) & (2) \\ \\ \hline
 &  &  \\
Pernambuco (2007 group) & -29.32** & -29.29** \\
 & (2.371) & (2.336) \\
 & [0.0230] & [0.0223] \\\\
Bahia and Paraíba (2011 group) & 2.552 & 2.995 \\
 & (3.904) & (3.857) \\
 & [0.570] & [0.523] \\\\
Ceará (2015 group) & 4.711 & 4.719 \\
 & (3.363) & (3.314) \\
 & [0.354] & [0.341] \\\\
Maranhão (2016 group) & -1.956 & -2.152 \\
 & (3.451) & (3.443) \\
 & [0.623] & [0.599] \\\\
 &  &  \\
\hline
Controls & No & Yes \\
Municipality FE & Yes & Yes \\
Year FE & Yes & Yes \\
Observations & 35,662 & 35,662 \\
R-squared & 0.699 & 0.700 \\
F-statistic & 206.6 & 196.4 \\
P-value & 4.19e-08 & 5.12e-08 \\
\hline\hline
\end{tabular}
\end{table}
\end{document}